\documentclass[10pt,letterpaper]{IEEEtran}

\usepackage{cite}
\usepackage[cmex10]{amsmath}
\usepackage{stfloats}
\usepackage{url}
\usepackage{graphicx}

\begin{document}
\title{Modeling Poly-Phase Capacitors in an Arbitrary Reference Frame}
\author{Adam Shea}

\maketitle

\begin{abstract}
\end{abstract}

\section{Introduction}
Poly-phase systems are commonly used in power conversion applications in order
to take advantage of the improved power handling capabilities for a given
voltage and current limit and the ability to make a rotating flux.


\section{Complex Vector Formulation}
For this paper we use the complex vector formulation described in
\cite{Novotny} which maps an arbitrary balanced poly-phase system into the
complex plane.
Equation \ref{eqPark} shows how this is done for three phases.
\begin{equation}
\label{eqPark}
f_{dq} = \frac{2}{3} \left({\alpha_a f_a + \alpha_b f_b + \alpha_c f_c}\right)
\end{equation}
The factor of $2 / 3$ causes the magnitude of the complex vector to be the
same as the magnitude of any of the terminal quantities.
This choice requires a factor of $3 / 2$ in order perform power calculations.
Other options for this factor cause other invariants when performing system
calculations.
The coefficients $\alpha_n$ are the position of the phases in the given
reference frame.
This can be used with differing $\alpha_n$ on the same phase quantities in
order to work in reference frames at different spatial and temporal
frequencies. \cite{Rockhill}
From this complex vector formulation, we can multiply by a fixed-speed
rotating unit vector to obtain complex quantities in a rotating reference
frame.

\section{Capacitor Model}
In this paper we are working with a three-phase star-connected capacitor with
terminal quantities defined in the standard passive convention (voltages
referenced phase-to-neutral, currents flowing into the phase terminals). 
The capacitor voltage differential equations are then
\begin{align}
\frac{d V_{an}}{dt} &= \frac{I_a}{C} \\
\frac{d V_{bn}}{dt} &= \frac{I_b}{C} \\
\frac{d V_{cn}}{dt} &= \frac{I_c}{C}
\end{align}

\section{Conclusions}

\end{document}
