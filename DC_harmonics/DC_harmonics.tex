\section{Introduction}
Investigating harmonic contributions of oscillator tolerance (~50ppm) would be
difficult if not intractable using standard FFT-based methods due to the
requirement to simulate seconds of data at MHz sampling rates in order to get
acceptable resolution.
To cope with this, we compute a closed form of the DC-link current spectral content.
DC-link capacitor sizing is driven by the harmonic content of the DC-link current.
Usually various standins are used such as ripple voltage or a rule of thumb
factor based on converter output current, but these crude methods make
significant assumptions based on electrolytic capacitor current handling and
standard-3-phase inverter or rectifier topologies.
As DC-link capacitor sizing dominates converter size, optimal capacitor choice is important.

\section{Teh Maths}
Closed form calculation of the harmonics of asymmetrical sine-triangle PWM for
a single half-bridge we start by parameterizing the top-side DC-link current
as a function of two variables $x=\omega_{c} t$ and $y=\omega_{0} t$ being the
carrier and modulation per-unit phase respectively.
This gives us a 2D function of current as shown in \autoref{fig:current_cell}.
This function can be defined as a piecewise surface
\autoref{eq:current}.

\begin{equation}
\label{eq:current}
I_{ph} (t) = 
	\begin{cases}
	I_pk \cos \left({\omega_{0} t + \theta_{pf}}\right) 
	& \text{if } 1 + M \cos\left({\omega_{0} t}\right)
	\leq t \leq 1 - M \cos\left({\omega_{0} t}\right),
	\\
	0 & \text{else}
	\end{cases}
\end{equation}

\subsection{Fourier Integral}
We then can calculate the 

\section{Conclusion}
The conclusion goes here.

% An example of a floating figure using the graphicx package.
% Note that \label must occur AFTER (or within) \caption.
% For figures, \caption should occur after the \includegraphics.
% Note that IEEEtran v1.7 and later has special internal code that
% is designed to preserve the operation of \label within \caption
% even when the captionsoff option is in effect. However, because
% of issues like this, it may be the safest practice to put all your
% \label just after \caption rather than within \caption{}.
%
% Reminder: the "draftcls" or "draftclsnofoot", not "draft", class
% option should be used if it is desired that the figures are to be
% displayed while in draft mode.
%
%\begin{figure}[!t]
%\centering
%\includegraphics[width=2.5in]{myfigure}
% where an .eps filename suffix will be assumed under latex, 
% and a .pdf suffix will be assumed for pdflatex; or what has been declared
% via \DeclareGraphicsExtensions.
%\caption{Simulation Results}
%\label{fig_sim}
%\end{figure}

% Note that IEEE typically puts floats only at the top, even when this
% results in a large percentage of a column being occupied by floats.


% An example of a double column floating figure using two subfigures.
% (The subfig.sty package must be loaded for this to work.)
% The subfigure \label commands are set within each subfloat command, the
% \label for the overall figure must come after \caption.
% \hfil must be used as a separator to get equal spacing.
% The subfigure.sty package works much the same way, except \subfigure is
% used instead of \subfloat.
%
%\begin{figure*}[!t]
%\centerline{\subfloat[Case I]\includegraphics[width=2.5in]{subfigcase1}%
%\label{fig_first_case}}
%\hfil
%\subfloat[Case II]{\includegraphics[width=2.5in]{subfigcase2}%
%\label{fig_second_case}}}
%\caption{Simulation results}
%\label{fig_sim}
%\end{figure*}
%
% Note that often IEEE papers with subfigures do not employ subfigure
% captions (using the optional argument to \subfloat), but instead will
% reference/describe all of them (a), (b), etc., within the main caption.


% An example of a floating table. Note that, for IEEE style tables, the 
% \caption command should come BEFORE the table. Table text will default to
% \footnotesize as IEEE normally uses this smaller font for tables.
% The \label must come after \caption as always.
%
%\begin{table}[!t]
%% increase table row spacing, adjust to taste
%\renewcommand{\arraystretch}{1.3}
% if using array.sty, it might be a good idea to tweak the value of
% \extrarowheight as needed to properly center the text within the cells
%\caption{An Example of a Table}
%\label{table_example}
%\centering
%% Some packages, such as MDW tools, offer better commands for making tables
%% than the plain LaTeX2e tabular which is used here.
%\begin{tabular}{|c||c|}
%\hline
%One & Two\\
%\hline
%Three & Four\\
%\hline
%\end{tabular}
%\end{table}


% Note that IEEE does not put floats in the very first column - or typically
% anywhere on the first page for that matter. Also, in-text middle ("here")
% positioning is not used. Most IEEE journals/conferences use top floats
% exclusively. Note that, LaTeX2e, unlike IEEE journals/conferences, places
% footnotes above bottom floats. This can be corrected via the \fnbelowfloat
% command of the stfloats package.


