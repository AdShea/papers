\chapter{Abstract Design}

\section{Control Topology}


\subsection{Non-synchronized PWM}
While control states may be managed between modular drive processors with only
moderate difficulty, the classical assumption that all PWM carriers are
synchronized presents significant difficulty as a delay of only a few
microseconds represents a significant phase delay.
Thankfully, at normal carrier to fundamental ratios, the phase of the PWM
carrier is unimportant to a first order as the machine inductance filters out
the high frequency switching components.
Quartz crystal-based main oscillators are quite stable and allow frequency
tolerance below 100ppm without any special calibration.
For a 20kHz PWM system, the beat frequency between PWM carriers will be at a
frequency of 2Hz.
This beat frequency will manifest as a peaking of the $\frac{dV}{dt}$ applied
to the machine windings relative to the frame every 0.5s.
While this phenomenon has little effect on the control of the machine, the
effects on bearing currents and insulation lifetime have not been investigated.

\subsection{Fault Management}

\section{DC-Link Design}

\subsection{Interconnect Design}

\subsection{Capacitor Sizing}

\section{Peak Ratings}

Electric machine and drive system ratings are influenced by a variety of
factors.
While continuous ratings for rated operating conditions are well defined and
easily modeled and tested, peak ratings are dependent on definitions.
The determination of peak ratings is further complicated by marketing
influences in that a large "peak power" number is an easily quoted figure of
merit even if it has little real-world backing.

The peak rating of an electric drive system is driven by a number of factors
depending on the peak time required.
For peak times on the order of one electrical cycle, the power limit is driven
by the saturation current allowed by the power switches and the
demagnetization limit of the permanent magnets in the electric machine.
Both of these limits are effected by the starting temperature of the
electrical drive system, but are largely independent of other thermal effects
as the peak time is too short.


