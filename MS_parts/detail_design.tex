\chapter{Detailed Design}
In this chapter I describe in detail the design of the modular motor drive
modules and system.
In the sections that follow, the specific design choices and design drivers
are outlined and schematic sections are presented.
Overall drive component schematics and PCB layouts are contained in Appendix
\ref{appCAD}.

\section{Mechanical Layout}
An important part of any integrated converter design is the mechanical layout
and packaging or as some would say: "multiphysics integration".
For this integrated modular drive design, the converter and all its controls
will be mounted on the end of a 6-phase machine designed at 1/3 scale but full
diameter for the FreedomCar/USDrive specification.
This layout is shown in rendered view in Fig. \ref{figRender}
\begin{figure}[htpb]
	\centering
	\label{figRender}
	\includegraphics[trim=16cm 2cm 8cm
	6cm,clip,width=\textwidth]{\detokenize{10kW_triview_w_ns}}
	\caption{Rendered view of modular drive on machine.}
\end{figure}

In order to shield the drive from EMI and fringing fields from the machine,
the drive is mounted on a stainless steel baseplate of the same footprint as
the stator.
The drive consists of six phase drive modules which mount over the
corresponding two stator tooth section of the machine.
Each module is one sixth of the stator circumference.
In order to allow easy fabrication each module consists of two printed circuit
boards with bolt together interconnects which both allows for low-resistance
connection for high-current paths and the ability to quickly change out
damaged modules in case of faults.

Cooling is designed to be possible with either water or air.
In this case water cooling is used as the machine is already cooled with a
water jacket.
The drive modules plumbed in series, upstream of the machine cooling water
jacket.

\section{Gate Drivers}
Gate drive design for a tightly integrated motor drive is primarily driven by
a need for small size and high output current.
In this design, the DC-Link voltage is less than 600V which allows the use of
commercial integrated bootstrap gate driver chips.

The FAN7390 integrated bootstrap gate driver was chosen as it supplies peak
switching currents of more than 4 amps and handles undervoltage lockouts to
ensure safe startup.
This device is supplied in an 8 pin SOIC package and requires an external
diode and bootstrap capacitor for high side supply.
The temperature rating is for less than 150C junction temperature which
requires less than 300mW of power dissipation in order to allow full
temperature range operation in this design.
Gate turn on and turn off is controlled through the use of a 10 Ohm turn on
resistor with a turn off diode to help guard against shoot-through and ensure
that Miller current injection cannot turn on the IGBTs.

\subsection{Desaturation Detection}
Power stage desaturation is measured on the low side by use of a diode clamped
connection to the switching node tied to the 15V gate supply.
This signal is then buffered by a P-channel MOSFET before being fed into a
digital input to the DSP as shown in figure \ref{figDesat}.
In the case of lower switch desaturation, this input will remain low which
indicates to the software that a fault has occurred.
In the case of high side desaturation, the fault can only be detected if the
switching node voltage drops below 15V.
This sensor can also detect shorted power switches within one switching cycle
due to state mismatch with the gate drive commands.

\begin{figure}[htbp]
\centering
\label{figDesat}
\includegraphics[page=3,trim=8.1cm 13.8cm 21.7cm 13.55cm,clip,width=\textwidth]{PCB}
\caption{Schematic of desaturation detection circuit. Left end connects to
switching node, right end to the DSP.}
\end{figure}

This method requires minimal hardware but will allow the controller to detect
a blown switch withing one switching cycle so long as the DC link voltage is
above 15V.


\section{Power Module}
While the power rating for this design is only 18kW, the system design was
implemented with higher power ratings in mind.
To this end, a small PCB was made to carry two discrete 600V IGBT-diode
CoPacks, an RTC temperature sensor and a phase current busbar.
This module mimics standard half-bridge IGBT modules from suppliers like
Powerex and Semikron at a smaller footprint and power rating more in line with
the design goals of this system.
The power module attaches to the control board by way of two bolt-down
connections for the DC-Link and an 8 pin 100 mil header for gate connections
and temperature sensing.
The phase conductor busbar is used as part of the current sensing system
described in the next section.
This module is then strapped to the heatsink using a fiberglass clamp to cool
the IGBTs.
The design here allows for either air or water cooling though water cooling
with a low cost video card water block was implemented during construction.

\section{Current Sensing}
Current sensing in this modular design was primarily driven by availability
and ease of integration.
While closed-loop sensors are most common for high performance motor drives,
their requirement of encircling the phase conductor to be sensed meant that
integration and easily disassembly would be difficult.
Giant Magnetoresistive (GMR) sensors were also considered as they provide very
small packages with excellent frequency response and dynamic range.
GMR sensors however are more immature in the current sensing space and the
unipolar nature of the sensor means that significant support is needed in
terms of biasing and signal processing.
The lack of a polished current sensing solution based on GMRs ruled them out
of this design.

The current sensing design for the integrated modular motor drive was finally
chosen to be an open-loop hall effect field sensor.
The FHS 40-P device from LEM is a small SOIC-8 packaged device with minimal
signal conditioning circuitry which is both low-cost and readily available.
In order to use this device, it must be mounted on an area of PCB with an open
aperture in the ground plane beneath it as is shown in figure
\ref{figHallPCB}.
The current in a conductor mounted a fixed distance away is then sensed by
measuring the magnetic field impinging on the sensor.

\begin{figure}[htbp]
	\centering
	\label{figHallPCB}
	\includegraphics[width=\textwidth]{\detokenize{20110926_003}}
	\caption{Side view of modular converter phase showing current sensor,
	waterblock, and phase conductor layout.}
\end{figure}
The spacing between the sensor and current carrying conductor is managed by
the bolt-together standoffs between the controller board and the power
semiconductor module.

Using a free-space field sensor in an application tightly integrated with an
electric machine presents special challenges which were only exacerbated by
the rotor in this machine being three times the length of the stator.
In order to ensure current sensing was not contaminated by offsets and noise
from the machine below, the baseplate was chosen of a semi-ferromagnetic
material (304 Stainless Steel) in order to shield low-frequency magnetic
fields from the sensor.
The copper faceplate of the water block directly below the power semiconductor
module adds a further layer of shielding, using eddy currents to block higher
frequency interfering fields from the sensor.

\section{DSP}
DSP selection for this converter was primarily driven by thermal
considerations. In to allow the possibility of operating with incoming coolant
at 105C, the processor needed to be able to operate at 125C or greater ambient
temperatures.

At the time of design the only two options readily available were
TMS320-series aerospace grade 200C processors from Texas Instruments or 140C
rated dsPIC processors from Microchip. As this application requires
approximately the same processing power as a standard three phase motor
controller, the motor control dsPIC series processor is more than adequate.

A benefit of this DSP is that it is available in a QFN package which is
extremely helpful for the tight packaging requirements of this design.
Furthermore, the Microchip DSP costs an order of magnitude less than the Texas
Instruments part primarily due to it's roots in automotive and industrial
controls rather than military and aerospace applications.

\section{Communications}
Inter-module communications is one of most important variables in an
integrated modular motor drive design. The minimum or optimal values for
for intermodule communication are an item for future work. For this design, a
combination of digital and analog communication was chosen.

In order to allow reasonable sensing of the current vector, each module
receives an analog current signal from two other modules.  One is another
phase on the same star-connection in order to allow calculation of the
complete current vector; the other is from an adjacent phase on the other
star-connection which is used for fault-detection and monitoring.
This connection was received through a differential amplifier with
low-impedance termination possible if noise pickup becomes a problem.  The
schematic of this receive amplifier is shown in Fig. \ref{figRecAmp}.
\begin{figure}[htbp]
	\centering
	\label{figRecAmp}
	\includegraphics[page=2,trim=3cm 11cm 21.2cm 12cm,clip,width=\textwidth]{PCB}
	\caption{Left-adjacent current signal receive amplifier.}
\end{figure}

This analog communication is supplemented by a CAN bus operating at 1Mbps.
This digital bus is used both for current vector command input and in order to
share lower-bandwidth trimming values between modules.

Each module outputs it's measured values for the three currents which is read
by the other modules and used to close the loop on sensor gain and offset in
order to make the sensor input for control operation identical for all phases.
