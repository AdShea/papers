\chapter{Introduction}
The increasing proliferation and maturation of electric drive technologies has
caused drive system-level integration to become a high priority in a variety
of application areas including electric propulsion, aerospace, white goods,
and down-hole drilling.
Tight integration of motor and drive electronics
offers attractive properties including reduced overall system volume,
reduction of high-current cabling, reduced radiated EMI, simplified cooling
arrangements, and appealing fault tolerance opportunities.

However, physical integration of electric machines and drives presents many
challenges.
In order to integrate the controls and power electronics into the
machine housing, the power electronics must be designed to operate reliably in
elevated vibration and thermal environments that fall outside of standard
ranges for industrial-grade components.
Further complicating the
challenges, customers for integrated drives typically expect the drive
electronics to meet or exceed the lifetime of the machine, requiring very high
reliability in demanding environments.

Early generations of integrated motor drives have typically used conventional
three-phase induction or PM synchronous machines combined with standard
voltage-source 6-switch bridge inverters to excite the machine.
In many of
these units, the drive is housed in its own enclosure which, in turn, is
mounted on the side or end of the machine .
More aggressive concepts for
integrated machine drives have been proposed that approach more closely the
ultimate objective of mounting the complete drive inside the same enclosure as
the machine ,, but commercialization of these advanced integrated drive
architectures has been limited to date.

One of these advanced concepts that has been proposed in the literature is the
integrated modular motor drive (IMMD) .
As illustrated in Fig. 1, this
concept segments the stator into individual pole pieces, each encircled by a
concentrated winding that is excited by its own dedicated electronics unit
that includes both the necessary power electronics and controls to form a
complete pole-drive unit.
The IMMD is completed by interconnecting a number
of these pole-drive units to form an annulus around the rotor.
Appealing features of this IMMD concept emerge from its modularity, offering
opportunities for improved manufacturability, enhanced fault tolerance and
redundancy, and simplified repair/replacement of the pole-drive units.

\section{Motivation}

\subsection{Reliability and Redundancy}

\subsection{EMI Containment}

\section{Previous Research}
